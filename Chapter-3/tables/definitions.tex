\begin{table}[p!]
\centering
\begin{center}
\caption{Definition and examples of the politeness peer interaction characteristic}
    \begin{tabular}{ |l|l|p{10cm}| }
		\hline
	\multicolumn{3}{ |c| }{\textbf{Politeness Criteria}} \\
	\hline
	\multirow{3}{*}{Tact} 
	 & Definition & Minimize cost and maximize benefit to peer \\
	 & Polite & \textit{``We can do all of it together, just sort by level.''} (S9) \\
	 & Impolite & \textit{``We can do a histogram...which is always sort of a pain in the butt to do in Excel.''} (L14) \\ \hline
	\multirow{3}{*}{Generosity} 
	 & Definition & Minimize benefit and maximize cost to self \\
	 & Polite & \textit{``CONCATENATE you can do. I can do this for
you, very easily.''} {S10} \\
	 & Impolite & \textit{``Maybe you should write a python script for this.''} {L6} \\ \hline
	\multirow{3}{*}{Approbation} 
	 & Definition & Minimize dispraise and maximize praise of peer \\
	 & Polite & \textit{``I'm not as good at the Excel stuff as you are.''} (L5) \\
	 & Impolite & \textit{``This[partner's suggestion] is useless.''} (S14) \\ \hline
	\multirow{3}{*}{Modesty} 
	 & Definition & Minimize praise and maximize dispraise of self \\
	 & Polite & \textit{``From whatever limited knowledge of data analysis I have, I think you need to create a linear regression model...''} (S14) \\
	 & Impolite & \textit{``I'm very good at Paint.''} (S10) \\ \hline
	\multirow{3}{*}{Agreement} 
	 & Definition & Minimize disagreement and maximize agreement between peers \\
	 & Polite & \textit{``Do you want to use Python?''} (S8) \\
	 & Impolite & \textit{``No, no, no...Don't you want it comma separated? That's what I'm doing.''} (S14) \\ \hline
	\multirow{3}{*}{Sympathy} 
	 & Definition & Minimize antipathy and maximize sympathy between peers\\
	 & Polite & \textit{``We can try JMP...'' [``I haven't done anything in JMP.''] ``Neither have I!''} (L14) \\
	 & Impolite & \textit{``It doesn't matter how you do it.''} (L16) \\ \hline
\end{tabular}
\label{tab:politeDef}
\end{center}
\end{table}

\begin{table}[p!]
\centering
\begin{center}
\caption{Definition and examples of the persuasiveness peer interaction characteristic}
    \begin{tabular}{ |l|l|p{10cm}| }
		\hline
	\multicolumn{3}{ |c| }{\textbf{Persuasiveness Criteria}} \\
	\hline
	\multirow{3}{*}{Content} 
	 & Definition & Recommender provides credible sources to verify use of the tool\\
	 & Persuasive & \textit{``Go here, go to Data. Highlight that...Data, Sort, and it lets you pick two.''} (L8) \\
	 & Unpersuasive & \textit{``Let's try to text filter, right?''} (S5) \\ \hline
	\multirow{3}{*}{Structure} 
	 & Definition & Messages are organized by climax-anticlimax order of arguments and conclusion explicitness \\
	 & Persuasive & \textit{``I know that SUMIF is a type of function that allows you to combine the capabilities of SUM over a range with a condition that needs to be met.''} (S3)\\
	 & Unpersuasive & \textit{``There's a thing on Excel where you can do that, where you can say if it is this value, include, if it is not, exclude...Yeah, IF.''} (S11) \\ \hline
	\multirow{3}{*}{Style} 
	 & Definition & Messages should avoid hedging, hesitating, questioning intonations, and powerless language \\
	 & Persuasive & \textit{``Control-Shift-End''} (S1) \\
	 & Unpersuasive & \textit{``I guess we're going to have to use some math calculations here, or a pivot table.''} (L9) \\ \hline
\end{tabular}
\label{tab:persuasiveDef}
\end{center}
\end{table}

\begin{table}[p!]
\centering
\begin{center}
\caption{Definition and examples of the receptiveness peer interaction characteristic}
    \begin{tabular}{ |l|l|p{9cm}| }
		\hline
	\multicolumn{3}{ |c| }{\textbf{Receptiveness Criteria}} \\
	\hline
	\multirow{3}{*}{Demonstrate Desire} 
	 & Definition & User showed interest in discovering, using, or learning more information about the suggested tool \\
	 & Receptive & \textit{``That was cool, how [the column] just populated.''} (S4) \\
	 & Unreceptive & \textit{[``So you want to use R for it?''] ``No, no, no...''} (S14) \\ \hline
	\multirow{3}{*}{Familiarity} 
	 & Definition & User explicitly expresses familiarity with the environment \\
	 & Receptive & \textit{``Control shift...how do I select it completely?''} (S2) \\
	 & Unreceptive & \textit{``I've never done anything in JMP.''} (L10) \\ \hline
\end{tabular}
\label{tab:receptiveDef}
\end{center}
\end{table}

\begin{table}[p!]
\centering
\begin{center}
\caption{Definition and examples of the time pressure peer interaction characteristic}
    \begin{tabular}{ |l|l|p{9cm}| }
		\hline
	\multicolumn{3}{ |c| }{\textbf{Time Pressure Criteria}} \\
	\hline
	\multirow{3}{*}{Time Pressure} 
	 & Definition & Participant makes statement regarding time after a recommendation \\
	 & Yes & ``\textit{Yeah, that would work if we had time.}'' (L5) \\
	 & No & No comments about time \\ \hline
\end{tabular}
\label{tab:timeDef}
\end{center}
\end{table}


\begin{table*}[ht]
\centering
\begin{center}
\caption{Definition and examples of the tool observability peer interaction characteristic}
    \begin{tabular}{ |l|l|p{9cm}| }
	\hline
	\multicolumn{3}{ |c| }{\textbf{Tool Observability Criteria}} \\
	\hline
	\multirow{3}{*}{Observability} 
	 & Definition & The ability to view the recommended tool through a graphical user interface \\
	 & Observable & ``\textit{Let's deploy a histogram}...[In Menu] \textit{Insert, Recommended Charts...}'' (S7) \\
	 & Non-Observable &  ``\textit{Control-Shift-End}'' (S1) \\ \hline
\end{tabular}
\label{tab:toolDef}
\end{center}
\end{table*}
