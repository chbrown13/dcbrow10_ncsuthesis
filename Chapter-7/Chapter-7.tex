\chapter{CONCLUSION}
\label{chap-end}


\section{Thesis Statement Revisited}

This dissertation presents research to evaluate and support my thesis statement (Chapter~\ref{chap-thesis}). The thesis of this dissertation is: \\

\thesis \\ \\

To support this claim, my research makes the several contributions to advance knowledge on designing automated recommendations to increase adoption of developer behaviors. First, I conducted a \textit{set of experiments} examining \textit{peer interactions} and the \tele as developer recommendation approaches to determine what makes an effective recommendation to software engineers (Chapter~\ref{chap-peer}). The results from these studies show that user-to-user recommendations are effective because of their ability to foster receptiveness, specifically desire and familiarity, while naive bots are ineffective because of their inability to conform to social context and development workflows.

The preliminary studies motivated the need for new techniques to recommend developer behaviors to programmers as opportunities for peer interactions decline and simple systems such as \toolone generate intrusive and unsuccessful notifications. To improve the effectiveness of automated recommendations, I introduce \framework, a \textit{conceptual framework} for designing effective automated recommendations to developers by applying concepts from nudge theory (Chapter~\ref{chap-framework}). This framework utilizes practical tools for choice architecture~\cite{johnson2012beyond} to obtain principles for creating automated recommendations to improve the environment surrounding developers' decisions, and the formative evaluation shows software engineers preferred actionable suggestions compared to static ones. 

To evaluate \framework, I devised a \textit{set of experiments} analyzing an existing recommender system, \suggs, through the lens of this framework to discover its impact on developer preferences and development activity (Chapter~\ref{chap-suggs}). These evaluations show that developers significantly prefer to receive tool recommendations from systems incorporating \framework than those that don't, and that this framework is useful for recommending a variety of changes and effective for improving development activity and collaboration between developers during reviews. 

Finally, to further assess this framework, I developed a novel \textit{automated recommender system}, \tooltwo, that incorporates \framework principles to generate digital nudges recommending useful developer behaviors to Computer Science students (Chapter~\ref{chap-bot}). The findings show that this bot was able to improve code quality and the productivity of students on their programming assignments by encouraging them to follow software engineering processes, a behavior that is also often ignored by professional developers in industry.

In this work, I use concepts from nudge theory to encourage software engineers to adopt better behaviors. I analyze existing and novel recommendation techniques, define \framework as a method to enhance automated recommendations, investigate how this framework impacts existing recommender systems, and develop a bot to show this approach influences programmer behavior by improving code quality and developer productivity.

\section{Future Work}

This dissertation motivates, presents, and evaluates \framework, a novel framework that incorporates nudge theory to design effective automated recommendations to encourage developer behaviors. Future directions of this research can further enhance automated developer recommendations by \textit{analyzing the behavior} of software engineers and \textit{developing new tools} and techniques to improve the decision-making, behavior, and productivity of programmers in their work.

\subsection{Behavior}

Examples of future studies related to this work involve using \framework to suggest additional developer behaviors, predicting the actions of programmers to proactively make recommendations to prevent bad behaviors, and exploring other disciplines to improve the behavior and decision-making of software engineers.

\begin{itemize}
    \item \textbf{Recommending developer behaviors.} The research presented in this dissertation explores recommendations for several developer behaviors, namely tool adoption (Chapter~\ref{chap-peer}, Chapter~\ref{chap-suggs}.1), code improvements (Chapter~\ref{chap-suggs}.2), and following software engineering processes (Chapter~\ref{chap-bot}). Future directions of this work can examine the following question: how can \framework impact the adoption of other developer behaviors? For example, research shows software engineers often ignore beneficial development practices such as pair programming~\cite{Lui2010}, software dependency updates~\cite{Samim2017AutoPullRequests}, software migration~\cite{SmartSheet}, and more. Future work can explore using the framework presented in this dissertation to design developer recommendations for additional tools and practices.
    
    \item \textbf{Predicting developer behavior.} For the most part, my research is largely reactive in that it makes recommendations to suggest developer behaviors after programmers have completed a programming task inefficiently. To improve the effectiveness of automated recommendations to developers, future work can explore developing proactive nudges to predict the actions of developer and present suggestions before bad behaviors occur. For example, machine learning techniques such as collaborative filtering~\cite{Murphy-Hill2012Fluency} or Bayesian user modeling~\cite{HorvitzLumiere} can be applied to analyze previous development activities of software engineers and anticipate poor decisions in advance. Then, recommender bots can proactively suggest better practices to help developers avoid poor practices and increase adoption of beneficial behaviors.
    
    \item \textbf{Interdisciplinary behavioral concepts.} Nudge theory is a behavioral science concept for improving human decision-making and behavior. To advance this research, future work can explore techniques for modifying human behavior from other disciplines. For example, behavioral science also posits \textit{shoves} as an alternative to nudges that force humans to adopt target behaviors~\cite{Shove}. Likewise, user experience and human factors research submits \textit{dark patterns}, or deceptive user interface designs, as another form of indirect influence to alter user behavior online.\footnote{\url{https://darkpatterns.org/index.html}} Prospective studies can explore multidisciplinary techniques to influence human behavior and apply these concepts to influence the behavior of software engineers.
\end{itemize}

\subsection{Tools}

Potential advancements of this research also include subsequent studies examining using \framework to improve the adoption of systems produced by research in industry, enhance the output of development tools, and creating new bots to encourage the adoption of developer behaviors.

\begin{itemize}
    \item \textbf{Improving research products.} As mentioned in Section 3.1.1, there are a variety of factors that contribute to the developer behavior adoption problem. While the research presented in this dissertation primarily focuses on improving the decision-making of software engineers, future work can investigate ways to improve other barriers to the adoption of development tools and practices. For example, studies show the development products and tools developed by researchers are often ineffective for industry practitioners~\cite{norman2010research,wohlin2013empirical}. Future research can explore ways to bridge the research-practice gap and improve the adoption of developer behaviors by convincing researchers and toolsmiths to develop and evaluate products relevant to software engineers that accommodate developer needs. By doing this, we can increase the awareness of software engineering research, techniques, and findings and adoption of useful tools and practices in industry.
    
    \item \textbf{Tool output.} This dissertation introduces \framework as a framework to design recommendations encouraging developers to adopt code fixes (\suggs) and software engineering processes (\tooltwo) to apply to their work. Another application of this research is to enhance how problems are presented to programmers. Research shows developers often ignore warnings for code smells, or potential problems within code~\cite{Yamashita2013CodeSmells}, and avoid static analysis tools due to incomprehensible output~\cite{Johnson2013Why}. Prior work has also explored ways to improve code smell notifications, including techniques to provide actionable static analysis alerts that mitigate false positives~\cite{Heckman2010Model}, lightweight visualizations to inspect smells during code reviews~\cite{Parnin2008CodeSmells}, ambient interactive designs to support identifying and refactoring code smells~\cite{MurphyHill2010CodeSmells}, and developer-driven code smell prioritization to rank bugs based on criticality~\cite{Pecorelli2020DDCodeSmells}. To reduce code smells and increase the quality of code, future work can build on this research by using \framework to design code smell notifications and nudge developers to fix reported issues and further encourage the adoption of useful tools and practices.

    \item \textbf{Nudge bots.} To further improve the behavior of developers, future work can develop automated tools incorporating \framework principles and concepts from nudge theory to make recommendations using different interventions. For example, this work examines recommendations on GitHub through automated pull requests (Chapter~\ref{chap-peer}.2), \sugg (Chapter~\ref{chap-suggs}), and automated issues (Chapter~\ref{chap-bot}). Future directions of this work can explore delivering recommendations to software engineers through similar techniques on other code hosting websites like GitLab\footnote{\url{https://about.gitlab.com/}} or BitBucket.\footnote{\url{https://bitbucket.org/}} Additionally, future work can recommend developer behaviors to software engineers through mechanisms evaluated in prior work such as StackOverflow posts~\cite{cai2019AnswerBot},\footnote{\url{https://stackoverflow.com/}} instant messages through Slack~\cite{lin2016slack},\footnote{\url{https://slack.com/}} posts to social media~\cite{begel2010social} or blog sites~\cite{barik2015heart},\footnote{\url{https://news.ycombinator.com/}} and other online programming communities. Furthermore, research can also explore other interventions such as chatbots~\cite{cerezo2019building} or automated program repair techniques~\cite{monperrus2019Repairnator} to recommend developer behavior. These examples can provide further methods to incorporate \framework recommendations to encourage the adoption of beneficial behaviors by developers.
\end{itemize}

\section{Epilogue}

\begin{center}
%\begin{quote}
\textit{``I think the most interesting topic for software engineering research in the next ten years is, \\\textbf{`How do we get working programmers to actually adopt better practices?'}}''\footnote{\url{https://twitter.com/gvwilson/status/1142245508464795649?s=20}}
%\end{quote}
\end{center}
